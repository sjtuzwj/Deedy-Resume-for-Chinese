%%%%%%%%%%%%%%%%%%%%%%%%%%%%%%%%%%%%%%%
% Deedy - One Page Two Column Resume
% LaTeX Template
% Version 1.2 (16/9/2014)
%
% Original author:
% Debarghya Das (http://debarghyadas.com)
%
% Original repository:
% https://github.com/deedydas/Deedy-Resume
%
% IMPORTANT: THIS TEMPLATE NEEDS TO BE COMPILED WITH XeLaTeX
%
% This template uses several fonts not included with Windows/Linux by
% default. If you get compilation errors saying a font is missing, find the line
% on which the font is used and either change it to a font included with your
% operating system or comment the line out to use the default font.
% 
%%%%%%%%%%%%%%%%%%%%%%%%%%%%%%%%%%%%%%
% 
% TODO:
% 1. Integrate biber/bibtex for article citation under publications.
% 2. Figure out a smoother way for the document to flow onto the next page.
% 3. Add styling information for a "Projects/Hacks" section.
% 4. Add location/address information
% 5. Merge OpenFont and MacFonts as a single sty with options.
% 
%%%%%%%%%%%%%%%%%%%%%%%%%%%%%%%%%%%%%%
%
% CHANGELOG:
% v1.1:
% 1. Fixed several compilation bugs with \renewcommand
% 2. Got Open-source fonts (Windows/Linux support)
% 3. Added Last Updated
% 4. Move Title styling into .sty
% 5. Commented .sty file.
%
%%%%%%%%%%%%%%%%%%%%%%%%%%%%%%%%%%%%%%%
%
% Known Issues:
% 1. Overflows onto second page if any column's contents are more than the
% vertical limit
% 2. Hacky space on the first bullet point on the second column.
%
%%%%%%%%%%%%%%%%%%%%%%%%%%%%%%%%%%%%%%


\documentclass[]{deedy-resume-openfont}
\usepackage{fancyhdr}
    
\pagestyle{fancy}
\fancyhf{}
    
\begin{document}

%%%%%%%%%%%%%%%%%%%%%%%%%%%%%%%%%%%%%%
%
%     LAST UPDATED DATE
%
%%%%%%%%%%%%%%%%%%%%%%%%%%%%%%%%%%%%%%
\lastupdated

%%%%%%%%%%%%%%%%%%%%%%%%%%%%%%%%%%%%%%
%
%     TITLE NAME
%
%%%%%%%%%%%%%%%%%%%%%%%%%%%%%%%%%%%%%%
\namesection{朱}{文杰}{ \urlstyle{same}\href{mailto:myname@sjtu.edu.cn}{myname@sjtu.edu.cn} | 114514

%%%%%%%%%%%%%%%%%%%%%%%%%%%%%%%%%%%%%%
%
%     COLUMN ONE
%
%%%%%%%%%%%%%%%%%%%%%%%%%%%%%%%%%%%%%%

\begin{minipage}[t]{0.25\textwidth} 

%%%%%%%%%%%%%%%%%%%%%%%%%%%%%%%%%%%%%%
%     EDUCATION
%%%%%%%%%%%%%%%%%%%%%%%%%%%%%%%%%%%%%%

\section{教育经历} 
\sectionsep

\subsection{上海交通大学}
\descript{学士学位,主修软件工程}
\descript{系统软件方向}
\location{2017.09-2021.06}
\sectionsep

%%%%%%%%%%%%%%%%%%%%%%%%%%%%%%%%%%%%%%
%     LINKS
%%%%%%%%%%%%%%%%%%%%%%%%%%%%%%%%%%%%%%

\section{链接}
\sectionsep
Zhihu://  \href{https://zhuanlan.zhihu.com/c_1039197804748595200}{\bf 朝闻君} \\
(专栏关注300+,账号关注7000+) \\    
Github:// \href{https://github.com/sjtuzwj}{\bf sjtuzwj} \\
LinkedIn://  \href{https://www.linkedin.com/in/sjtuzwj}{\bf sjtuzwj} \\

%%%%%%%%%%%%%%%%%%%%%%%%%%%%%%%%%%%%%%
%     COURSEWORK
%%%%%%%%%%%%%%%%%%%%%%%%%%%%%%%%%%%%%%

 \section{相关修业}

\begin{tabular}{ll}
96  & 分布式系统 \\
91  & 操作系统 \\
99  & 计算机系统基础I上 \\
93  & 计算机系统基础I下 \\
91  & 计算机系统基础II \\
92  & 计算机系统工程 \\
95  & 程序设计与数据结构 \\
89  & 算法原理 \\
96  & 软件工程导论 \\
98  & 程序设计思想与方法 \\
92  & 离散数学 \\
92  & 项目管理及软件开发 \\
A+  & 可视化分析 \\
\end{tabular}
\sectionsep

%%%%%%%%%%%%%%%%%%%%%%%%%%%%%%%%%%%%%%
%     SKILLS
%%%%%%%%%%%%%%%%%%%%%%%%%%%%%%%%%%%%%%

\section{技能}
\sectionsep
\subsection{编程}
\location{主力}
C  \textbullet{} C++ \textbullet{} Java \\
\location{辅助}
Python \textbullet{}R \textbullet{} Assembly \\
\location{一般}
Typescript  \textbullet{} \LaTeX\\\ 

\sectionsep
\subsection{分布式}
\location{略懂}
Zookeeper \textbullet{} Dubbo \textbullet{} Springboot\\
\sectionsep

\sectionsep
\subsection{自动化测试}
\location{性能测试}
Load Runner \\
\location{功能测试}
JUnit \textbullet{} Selenium\\
\sectionsep
%%%%%%%%%%%%%%%%%%%%%%%%%%%%%%%%%%%%%%
%
%     COLUMN TWO
%
%%%%%%%%%%%%%%%%%%%%%%%%%%%%%%%%%%%%%%

\end{minipage} 
\hfill
\begin{minipage}[t]{0.73\textwidth} 

%%%%%%%%%%%%%%%%%%%%%%%%%%%%%%%%%%%%%%
%     EXPERIENCE
%%%%%%%%%%%%%%%%%%%%%%%%%%%%%%%%%%%%%%

\section{竞赛经历}
\sectionsep
\runsubsection{\href{https://www.zhihu.com/question/320759894/answer/656496386}{\bf 美国大学生数学建模竞赛}}
\descript{数据题M奖}
\location{2019.01 - 2019.01 | 远程}
\vspace{\topsep}
\begin{tightemize}
    \item 担任建模手与编程手,制定数据处理算法提供数据分析与数据可视化
    \item 通过巴斯扩散模型和场论建模,通过R语言进行探索性因子分析
    \item 撰写论文摘要

\end{tightemize}
\sectionsep
\runsubsection{\href{https://github.com/sjtuzwj/ChinaVis19-C1}{\bf ChinaVis挑战赛}}
\descript{组员}
\location{2019.04-2019.06 | 远程}
\begin{tightemize}
\item 项目最终获得优秀奖
\item 基于会议人员的GPS信息,分析时空轨迹
\item 特征工程提炼轨迹向量并按照空间尺度加权求和
\item 根据向量的相对编辑距离构建距离矩阵并进行聚类分析,进行二维可视化
\item 分析人员种类,并从中发掘异常情况
\end{tightemize}
\sectionsep

%%%%%%%%%%%%%%%%%%%%%%%%%%%%%%%%%%%%%%
%     RESEARCH
%%%%%%%%%%%%%%%%%%%%%%%%%%%%%%%%%%%%%%

\section{项目}
\sectionsep


\runsubsection{\href{https://github.com/sjtuzwj/DistributedKeyValueStorage}{\bf 分布式键值对存储系统}}
\descript{Java | 维护者}
\location{2020.6}
\begin{tightemize}
    \item 可伸缩的键值对分库存储系统
    \item 本科分布式课程最终实验,答辩获得A+评价
    \item 负载均衡使用一致性哈希+惰性迁移减少分库增加时数据迁移开销
    \end{tightemize}
\sectionsep

\runsubsection{\href{https://github.com/sjtuzwj/FMMM}{\bf 数据可视化与可视化分析}}
\descript{Python | 参与者}
\location{2019.6}
\begin{tightemize}
    \item Domain-specific的材料学数据聚类算法,材料性质可视化全栈
    \item 答辩获得A+评价
    \item 聚类目前正在申请发明专利,排行位于双方导师和研究生之后
	\item 专利申请公布号CN111178270A
    \end{tightemize}
\sectionsep


\runsubsection{\href{https://github.com/sjtuzwj/SEDA-Web-Http-Server}{\bf Socket Web服务器}}
\descript{Java | 维护者}
\location{2020.7}
\begin{tightemize}
    \item 仿造SEDA论文实现的web 服务器
    \item 主从Reactor + NIO
    \item 分阶段事件驱动模型+异步事件批处理
    \end{tightemize}
\sectionsep



\runsubsection{{\bf 操作系统课程实验}}
\descript{C/Assembly | 学生}
\location{2020.6}
\begin{tightemize}
    \item 实现IPADS实验室教学使用的Chcore微内核
    \item 填充关键函数,如伙伴系统,页表管理,IPC,调度
    \end{tightemize}
\sectionsep

\runsubsection{\href{https://github.com/sjtu-jiaojiao/SJTU-JiaoJiao}{\bf 交交二手交易信息平台}}
\descript{Typescript | 参与者}
\location{2019.9}
\begin{tightemize}
    \item 基于微服务的二手交易系统
    \item 本科暑期大作业,答辩获得A评价 \textbf{17 stars}
    \item 实现管理员前端的设计以及后台统计数据的可视化,使用Docker部署
    \end{tightemize}
\sectionsep

%%%%%%%%%%%%%%%%%%%%%%%%%%%%%%%%%%%%%%
%     OPEN SOURCE
%%%%%%%%%%%%%%%%%%%%%%%%%%%%%%%%%%%%%%

\section{开源}
\begin{tabular}{ll}
\href{https://github.com/sjtuzwj/RefactorMindMap}{\bf sjtuzwj/RefactorMindMap} & 重构:改善现有代码设计读书笔记思维导图 \\
\href{https://github.com/sjtuzwj/OperationalResearchModeling}{\bf sjtuzwj/OperationalResearchModeling} & 提供排队论的离散事件仿真程序,文章阅读量5000+\\
\href{https://github.com/SJTU-SE/awesome-se}{\bf SJTU-SE/awesome-se} & 提供软件学院课程实验的参考实现 \\
\end{tabular}
\sectionsep



\end{minipage} 
\end{document}  \documentclass[]{article}
